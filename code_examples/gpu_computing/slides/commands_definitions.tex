\newcommand{\code}[1]{\texttt{\smaller#1}}
\newcommand{\reals}{\mathbb{R}}
\newcommand{\complex}{\mathbb{C}}
\newcommand{\E}{\mathbf{E}}
\newcommand{\var}{\mathrm{Var}}
\newcommand{\cov}{\mathrm{Cov}}

\definecolor{mygreen}{rgb}{0,0.6,0}
\definecolor{mygray}{rgb}{0.5,0.5,0.5}
\definecolor{mymauve}{rgb}{0.58,0,0.82}
\definecolor{bggray}{rgb}{0.95,0.95,0.975}

\lstdefinestyle{prompt}{ %
	backgroundcolor=\color{bggray},
	basicstyle=\ttfamily\footnotesize,
	language=bash,
	frame=none,
	morekeywords={\$},
	keywordstyle=\ttfamily\sl,
	numbers=left,
	numberstyle=\ttfamily\tiny\color{mygray},
	autogobble=true
}

\lstdefinestyle{python_output}{ %
	backgroundcolor=\color{bggray},
	basicstyle=\ttfamily\footnotesize,
	language=bash,
	frame=none,
	keywordstyle=\color{black},
	numbers=left,
	numberstyle=\ttfamily\tiny\color{mygray},
	stringstyle=\color{black},
	autogobble=true
}

\lstset{ %
	backgroundcolor=\color{bggray},   % choose the background color; you must add \usepackage{color} or \usepackage{xcolor}; should come as last argument
	basicstyle=\ttfamily\footnotesize,        % the size of the fonts that are used for the code
	breakatwhitespace=false,         % sets if automatic breaks should only happen at whitespace
	breaklines=true,                 % sets automatic line breaking
	captionpos=b,                    % sets the caption-position to bottom
	commentstyle=\color{mygreen},    % comment style
	deletekeywords={...},            % if you want to delete keywords from the given language
	escapeinside={\%*}{*)},          % if you want to add LaTeX within your code
	extendedchars=true,              % lets you use non-ASCII characters; for 8-bits encodings only, does not work with UTF-8
	frame=none,                      % adds a frame around the code
	keepspaces=true,                 % keeps spaces in text, useful for keeping indentation of code (possibly needs columns=flexible)
	keywordstyle=\color{blue},       % keyword style
	language=Python,                 % the language of the code
	morekeywords={*,True,False,numpy,scipy,matplotlib,np,sc,plt,la,linalg,whos,det,inv,diag,array,allclose,linspace,pyplot,plt,show,subplots,plot,grid,legend,set_xlabel,set_ylabel,set_title,argmax,num,linewidth,color,label,rc,usetex,math,pandas,DataFrame,random,zeros,nanmax,nanargmax,nan,ones,polyfit,polyval,optimize,fsolve,interpolate,interp1d,stats,norm,diff,cdf,pdf,reshape,polynomial,hermite,hermgauss,sqrt,fftpack,as,sparse,spmatrix,jit,njit,cuda,vectorize,guvectorize,stencil,@jit,@njit,@cuda,@vectorize,@guvectorize,@stencil...},       % if you want to add more keywords to the set
	numbers=left,                    % where to put the line-numbers; possible values are (none, left, right)
	numbersep=5pt,                   % how far the line-numbers are from the code
	numberstyle=\ttfamily\tiny\color{mygray}, % the style that is used for the line-numbers
	rulecolor=\color{mygray},         % if not set, the frame-color may be changed on line-breaks within not-black text (e.g. comments (green here))
	showspaces=false,                % show spaces everywhere adding particular underscores; it overrides 'showstringspaces'
	showstringspaces=false,          % underline spaces within strings only
	showtabs=false,                  % show tabs within strings adding particular underscores
	stepnumber=1,                    % the step between two line-numbers. If it's 1, each line will be numbered
	stringstyle=\color{mymauve},     % string literal style
	tabsize=2,						 % sets default tabsize to 2 spaces
	autogobble=true%,	             % adjusts indentation and newline characters
	%title=\lstname                   % show the filename of files included with \lstinputlisting; also try caption instead of title
}
