\documentclass[10pt, aspectratio=1610, natbib, handout]{beamer}
\usepackage[utf8]{inputenc}
\usepackage{txfonts}
\usepackage{amsfonts,amssymb,amsmath}
\usepackage[T1]{fontenc}
\usepackage{appendixnumberbeamer,booktabs,tikz,pgfplots}
\pgfplotsset{compat=1.17}

\setbeamertemplate{navigation symbols}{%
    \usebeamerfont{footline}%
    \usebeamercolor[fg]{footline}%
    \hspace{1em}%
    \insertframenumber/\inserttotalframenumber
}

\hypersetup{
  colorlinks=true,
  linkcolor=black,
  urlcolor=blue,
  citecolor=black
}

\definecolor{dimgray}{RGB}{170, 170, 170}
\definecolor{alert}{RGB}{220, 0, 0}
\definecolor{main}{RGB}{51, 51, 179}
\definecolor{background}{RGB}{255, 255, 255}
\newcommand{\dimmer}[1]{\textcolor{dimgray}{#1}}
\newcommand{\gotobutton}[2]{\hyperlink{#1}{\beamergotobutton{#2}}}
\newcommand{\backbutton}[1]{\hyperlink{#1}{\beamerreturnbutton{Back}}}
\newcommand{\gototitlebutton}[2]{\hfill\gotobutton{#1}{#2}}
\newcommand{\backtitlebutton}[1]{\hfill\backbutton{#1}}
\setbeamercolor{button}{bg=main,fg=background}
\setbeamercolor{alerted text}{fg=alert}

% \newcommand{\mycite}[1]{\citeauthor{#1} (\citeyear{#1})}
\newcommand{\E}{\mathbf{E}}
\newcommand{\email}[1]{\href{mailto:#1}{#1}}
\newcommand{\website}[1]{\href{https://#1}{#1}}

\title[Huggett-Aiyagari]{
  \textbf{Bewley-type Models: Huggett \& Aiyagari}
}

\subtitle[Macro 3: TA\#5]{
  \textbf{Macroeconomics 3:} TA class \#5
}

\author[A.~Pasqualini]{
  Andrea Pasqualini
}

\institute[Bocconi]{Bocconi University}

\date{
  8 March 2021
}

\begin{document}

  \begin{frame}
    \maketitle
  \end{frame}

  \begin{frame}
    \frametitle{Plan for Today}

    Objective: \textbf{Replicate the results in \cite{Huggett1993} and \cite{Aiyagari1994}}

    \vfill\pause

    Sub-goals:
    \begin{itemize}
      \item Understand the papers
      \item Learn about ``binning'' (a.k.a., non-stochastic simulations)
      \item Learn about transition dynamics (a.k.a., MIT shocks)
    \end{itemize}

  \end{frame}

  \begin{frame}
    \frametitle{Overview}

    \cite{Huggett1993}
    \begin{itemize}
      \item Tries to address the Equity Premium Puzzle
      \item Uses a simple exchange economy with incomplete insurance
      \item Solid numerical approach: guaranteed to converge
      \item Finds that incomplete cannot account for the EPP
    \end{itemize}

    \vfill\pause

    \cite{Aiyagari1994}
    \begin{itemize}
      \item Tries to see if precautionary savings explain aggregate savings
      \item Uses an RBC economy with incomplete insurance
      \item Tentative numerical approach: converges, maybe, nobody knows why
      \item Finds that precautionary savings cannot account for aggregate savings
    \end{itemize}

  \end{frame}

  \begin{frame}
    \frametitle{Huggett: Overview}

    \begin{description}
      \item[Question] How to address the Equity Premium Puzzle (EPP) in macro models?
      \vfill\pause
      \item[Approach] A heterogeneous-agents model with incomplete insurance
      \vfill\pause
      \item[Challenge] Need to nail mechanism and quantification
      \vfill\pause
      \item[Findings] Preventing over-saving does it qualitatively, but not quantitatively
    \end{description}

  \end{frame}

  \begin{frame}
    \frametitle{Huggett: Methodology}

    \textbf{Equity Premium Puzzle:} spread between risk-free rate and risky rate too small in models relative to the data

    \vfill\pause

    Two avenues
    \begin{itemize}
      \item Risky rate too small in models
      \item Risk-free rate too large in models
    \end{itemize}

    \vfill\pause

    Huggett
    \begin{itemize}
      \item Risk-free rate is too large in models \hfill \dimmer{(at the time, innovative!)}
      \item Can reduce the risk-free rate by reducing the demand for the risk-free asset
      \item Two alternatives
        \begin{itemize}
          \item Prevent lenders from saving too much \hfill \dimmer{(unrealistic assumption)}
          \item Prevent borrowers from taking too much debt \hfill \dimmer{(more realistic)}
        \end{itemize}
      \item Representative-agent models cannot cut it: requires a zero net financial position in GE
    \end{itemize}

  \end{frame}

  \begin{frame}
    \frametitle{Huggett: Model}

    Ex-ante identical consumers solve the following
    \begin{align*}
      \max_{C_t, A_{t+1}} &\; \E_0 \left( \sum_{t=0}^{\infty} \beta^t \frac{C_t^{1-\gamma}}{1-\gamma} \right) \\
      \text{s.t.} &\;
      \begin{cases}
        C_t + A_{t+1} \leq Y_t + (1 + r_t) A_t & \forall\ t \\
        A_{t+1} \geq \underline{A} & \forall\ t \\
        \log(Y_{t+1}) = (1 - \rho) \mu + \rho \log(Y_t) + \varepsilon_{t+1} & \forall\ t \\
        \varepsilon_{t} \overset{iid}{\sim} \mathcal{N}(0, \sigma^2) & \forall\ t
      \end{cases}
    \end{align*}

    \vfill\pause

    The borrowing constraint $\underline{A}$ is such that it may bind for some consumers (i.e., $\underline{A}$ is higher than the natural debt limit)

  \end{frame}

  \begin{frame}
    \frametitle{Huggett: Numerical Approach}

    \begin{enumerate}
      \item At iteration $j$, guess an equilibrium interest rate $r^{(j)}$
      \vfill\pause
      \item Solve for the policy function $A'(A, Y)$
      \vfill\pause
      \item Combine $A'(A, Y)$ with $\Pi$ to obtain the endogenous transition matrix $Q$
      \vfill\pause
      \item Compute the ergodic distribution $\lambda(A, Y)$ by iterating $Q$ enough times
      \vfill\pause
      \item Compute the net excess demand $E^d(r) = \sum_A \sum_Y \lambda(A, Y)\ A'(A, Y)$
      \vfill\pause
      \item Use a root-solver to bring the LHS to the RHS (i.e., zero)
        \begin{itemize}
          \item If $E^d \left( r^{(j)} \right) > 0$, then $r^{(j)} < r^*$
          \item If $E^d \left( r^{(j)} \right) < 0$, then $r^{(j)} > r^*$
          \item Set $r^{(j+1)}$ accordingly and repeat 2--6
        \end{itemize}
    \end{enumerate}

  \end{frame}

  \begin{frame}
    \frametitle{Aiyagari: Overview}

    \begin{description}
      \item[Question] What are the determinants of aggregate savings in the US?
      \vfill\pause
      \item[Approach] A heterogeneous-agents RBC model
      \vfill\pause
      \item[Challenge] Lucas' diversification argument: idiosyncrasies average out in the aggregate
      \vfill\pause
      \item[Findings]
        \begin{itemize}
          \item Precautionary savings do not matter for aggregate savings, quantitatively
          \item Can generate precautionary savings without prudence
          \item Model matches the US cross-sectional distribution of household wealth
        \end{itemize}
    \end{description}

  \end{frame}

  \begin{frame}
    \frametitle{Aiyagari: Methodology}

    Fix aggregate (capital) savings $K$

    \vfill\pause

    Look at composition of the aggregate
    \begin{itemize}
      \item Thourough exploration of endogenous distribution of agents
      \item Focus on left tail of wealth distribution (constrained, or almost, agents)
    \end{itemize}

    \vfill\pause

    Features of the model
    \begin{itemize}
      \item Relatability: just an RBC model
      \item Source of heterogeneity: labor endowments
      \item No aggregate uncertainty: prices $r$ and $w$ fixed in equilibrium
    \end{itemize}

  \end{frame}

  \begin{frame}
    \frametitle{Aiyagari: Model}

    \begin{columns}[T]
      \begin{column}{0.45\textwidth}
        Households (demand side)
        \begin{align*}
          \max_{c_t, k_{t+1}} &\; \E_0 \left( \sum_{t=0}^{\infty} \beta^t\ \frac{c_t^{1-\gamma}}{1-\gamma} \right) \\
          \text{s.t.} &\;
          \begin{cases}
            c_t + k_{t+1} \leq w l_t + (1 + r) k_t & \forall\ t \\
            c_t, k_{t+1} \geq 0 & \forall\ t \\
            l_{t+1} = (1 - \rho) \mu + \rho l_t + \varepsilon_{t+1} & \forall\ t \\
            \varepsilon_t \overset{iid}{\sim} \mathcal{N}(0, \sigma^2) & \forall\ t
          \end{cases}
        \end{align*}
      \end{column}
      \begin{column}{0.45\textwidth}
        Firms (supply side) \\ \dimmer{(note: no time subscripts)}
        \begin{align*}
          \max_{K, L} &\; A K^\alpha L^{1-\alpha} - r K - w L
        \end{align*}
      \end{column}
    \end{columns}

    \vfill\pause

    Market clearing (recursive notation, ergodic equilibrium)
    \begin{equation*}
      K = \int_{k} \int_{l} \lambda(k, l)\ k'(k, l) \text{d} l\ \text{d} k
    \end{equation*}

  \end{frame}

  \begin{frame}
    \frametitle{Aiyagari: Numerical Approach}

    \begin{enumerate}
      \item At iteration $j$, guess an aggregate level of capital holdings $K^{(j)}$
      \vfill\pause
      \item Use the FOCs of the firm to compute $r$ and $w$
      \vfill\pause
      \item Solve for the households' policy function $k'(k, l)$
      \vfill\pause
      \item Combine $\Pi$ with $k'(k, l)$ to obtain the endogenous transition matrix $Q$
      \vfill\pause
      \item Compute the ergodic distribution $\lambda(k, l)$ by iterating $Q$ enough times
      \vfill\pause
      \item Compute aggregate savings $\hat{K}$ as
        \begin{equation*}
          \hat{K} \equiv \sum_{k} \sum_l \lambda(k, l)\ k'(k, l)
        \end{equation*}
      \vfill\pause
      \item Check if $\hat{K}$ is consistent with $K^{(j)}$
        \begin{itemize}
          \item If $\hat{K} \neq K^{(j)}$, then set $K^{(j+1)}$ using the dampening scheme for $\theta \in [0, 1]$
            \begin{equation*}
              K^{(j+1)} \equiv \theta \hat{K} + (1 - \theta) K^{(j)}
            \end{equation*}
          \item Repeat steps 2--7
        \end{itemize}
    \end{enumerate}

  \end{frame}

  \begin{frame}
    \frametitle{Are These Bewley-Type Models? Yes!}

    \begin{itemize}
      \item Consumers are ex-ante identical
        \begin{itemize}
          \item One maximization problem describes everybody (ex-ante!)
          \item Idiosyncratic uncertainty and policy functions place consumers differently on the distribution of asset holdings
        \end{itemize}

      \vfill\pause

      \item In equilibrium, consumers are heterogeneous
        \begin{itemize}
          \item At the ``dawn of time,'' there is a distribution of agents, depending on endowments
          \item Based on (different) income, consumers choose (different) savings
        \end{itemize}

      \vfill\pause

      \item At the \textit{ergodic} equilibrium
        \begin{itemize}
          \item Consumers are distributed according to an endogenous ergodic distribution
          \item The \textit{ergodic} distr.~does \textbf{not} mean that there is no dynamics (i.e., not a deterministic steady s.)
          \item Across periods, consumers are ``reshuffled'' such that the ergodic distribution is maintained
        \end{itemize}

      \vfill\pause

      \item Technical note
        \begin{itemize}
          \item There is idiosyncratic uncertainty
          \item There is \textbf{no} aggregate uncertainty
        \end{itemize}
    \end{itemize}

  \end{frame}

  \begin{frame}
    \frametitle{Comparing Huggett \& Aiyagari}

    They look like the same model\dots\ They are, almost

    \vfill\pause

    \begin{table}
      \centering
      \begin{tabular}{lll}
        \toprule
        & \textbf{Huggett} & \textbf{Aiyagari} \\
        \cmidrule{2-3}
        Supply side & Exogenous & Endogenous \\
        Financial market & Asset & Capital \\
        Research question & Equity Premium Puzzle & Composition of aggr.~savings \\
        Numerical solution & Net excess demand & Consistency w/~guess \\
        \bottomrule
      \end{tabular}
    \end{table}

    \vfill\pause

    \begin{itemize}
      \item Huggett is closer to a simple exchange economy
      \item Aiyagari is closer to the RBC model
      \item Huggett's numerical approach converges for sure
      \item Aiyagari's numerical approach converges\dots\ nobody knows why
    \end{itemize}

  \end{frame}

  \begin{frame}
    \frametitle{Practice Time}

    Moving to a Jupyter Notebook

  \end{frame}

  \appendix


  \begin{frame}
    \frametitle{References}

    \bibliographystyle{apalike}
    \bibliography{references}

  \end{frame}


\end{document}
