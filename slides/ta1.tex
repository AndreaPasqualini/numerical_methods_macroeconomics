\documentclass[10pt, aspectratio=1610]{beamer}
\usepackage[utf8]{inputenc}
\usepackage{txfonts}
% \usepackage{roboto}
\usepackage[T1]{fontenc}
\usepackage{appendixnumberbeamer}

\setbeamertemplate{navigation symbols}{%
    \usebeamerfont{footline}%
    \usebeamercolor[fg]{footline}%
    \hspace{1em}%
    \insertframenumber/\inserttotalframenumber
}

\hypersetup{
  colorlinks=true,
  linkcolor=black,
  urlcolor=blue,
  citecolor=black
}

\newcommand{\E}{\mathbf{E}}
\newcommand{\email}[1]{\href{mailto:#1}{#1}}
\newcommand{\website}[1]{\href{https://#1}{#1}}

\title[Intro to Numerical Methods for Macro]{
  \textbf{Introduction to Numerical Methods for Macroeconomics}
}

\subtitle[Macro 3: TA\#1]{
  \textbf{Macroeconomics 3:} TA class \#1
}

\author[A.~Pasqualini]{
  Andrea Pasqualini
}

\institute[Bocconi]{Bocconi University}

\date{
  8 February 2021
}

\begin{document}

  \begin{frame}
    \maketitle
  \end{frame}

  \begin{frame}
    \frametitle{About Myself}

    Hi, I am Andrea Pasqualini!

    \vfill\pause

    \begin{itemize}
      \item Graduated a couple of weeks ago! (yay!)
      \item Research interests: Banking, Macroeconomics
      \item JMP: markups on lending rates and markdowns on deposit rates
      \item Side project: Unemployment, SDFs and Dual Labor Markets in Europe
    \end{itemize}

    \vfill\pause

    % \begin{itemize}
    %   \item \textbf{Email:} \email{andrea.pasqualini@unibocconi.it}
    %   \item \textbf{Website:} \website{andrea.pasqualini.io}
    %   \item \textbf{Office:} 5-E2-FM02
    % \end{itemize}
    \begin{description}
      \item[Email] \email{andrea.pasqualini@unibocconi.it} (also for MS Teams)
      \item[Website] \website{andrea.pasqualini.io}
      \item[Office] 5-E2-FM02
      \item[Material] \href{https://github.com/AndreaPasqualini/numerical_methods_macroeconomics}{https://github.com/AndreaPasqualini/numerical\_methods\_macroeconomics}
    \end{description}

  \end{frame}

  \begin{frame}
    \frametitle{About the TA Classes for Macro 3}

    \begin{itemize}
      \item So far, you saw the theoretical tools in Macroeconomics
      \item Now it's time for fun: empirical tools!
    \end{itemize}

    \vfill\pause

    \begin{itemize}
      \item Solving Macro models analytically may be impossible (it is, very often)
      \item Need to obtain numerical solutions
      \item Two options
        \begin{itemize}
          \item Perturbation methods
          \item Projection methods
        \end{itemize}
    \end{itemize}

    \vfill\pause

    \begin{itemize}
      \item These TA classes: \textbf{projection methods}
      \item Applications: \textbf{macro models with heterogeneous agents}
    \end{itemize}

    \vfill\pause

    Objective: \textbf{get familiar with projection methods and related applications}

  \end{frame}

  \begin{frame}
    \frametitle{About the Tools}

    Objective: \textbf{manipulate numerical objects (e.g., matrices), plot results}

    \vfill\pause

    \begin{itemize}
      \item Many options available: Matlab, R, Python, etc.
      \item This course: Python
    \end{itemize}

    \vfill\pause

    Advantages of Python
    \begin{itemize}
      \item Free and open-source, reliable tool
      \item Unbeatable flexibility
    \end{itemize}

    \vfill\pause

    \begin{itemize}
      \item Many options to work with Python: VSCode, Spyder, PyCharm, Jupyter Notebooks, etc.
      \item These classes: Jupyter Notebooks (VSCode behind the scenes)
    \end{itemize}

  \end{frame}

  \begin{frame}
    \frametitle{Intro to Numerical Methods for Economists}

    Objective: \textbf{solve a model}

    \vfill\pause

    \begin{description}
      \item[What?] $\E_t \left( f \left( X_{t-1}, X_t, X_{t+1} \right) \right) = 0$ \hspace{2em} \textcolor{gray}{(only rational expectations)}
      \pause
      \item[Who?] Economists in Macroeconomics, Development Econ, Applied Micro
      \pause
      \item[Why?] Obtain predictions and counterfactuals, compare with data
      \pause
      \item[How?] Techniques based on a model's mathematical properties
      \pause
      \item[When?] All the effing time
    \end{description}

    \vfill\pause

    \begin{columns}
      \begin{column}{0.475\textwidth}
        \begin{block}{Perturbation methods}
          \begin{itemize}
            \item Rely on Taylor expansion
            \item Require differentiability of the model
            \item Low computational costs
          \end{itemize}
        \end{block}
      \end{column}
      \begin{column}{0.475\textwidth}
        \begin{block}{Projection methods}
          \begin{itemize}
            \item Rely on Bellman equations
            \item Allow for heterogeneity, discontinuities
            \item High computational costs
          \end{itemize}
        \end{block}
      \end{column}
    \end{columns}

  \end{frame}

  \begin{frame}
    \frametitle{Intro to Numerical Methods for Economists: Hands-on Example}

    Example: \textbf{Neoclassical Stochastic Growth Model}
    \begin{align*}
      \max_{C_t, K_{t+1}} &\; \E_0 \sum_{t=0}^{\infty} \beta^t u(C_t) \\
      \text{s.t.} &\; \begin{cases}
        C_t + K_{t+1} = Z_t K_t^\alpha + (1 + \delta) K_{t-1} &\forall\ t\\
        \log(Z_{t+1}) = (1-\rho) \mu + \rho \log(Z_t) + \log(\varepsilon_{t+1}) &\forall\ t\\
        \varepsilon_{t+1} \overset{iid}{\sim} \mathcal{N}(0, \sigma^2) &\forall\ t\\
        C_t, K_{t+1} > 0 &\forall\ t\\
        K_{0}, Z_{0} \text{ given}
      \end{cases}
    \end{align*}

    \vfill\pause

    \begin{columns}[T]
      \begin{column}{0.475\textwidth}
        Variables
        \begin{itemize}
          \item Endogenously predetermined: $Z_t$, $K_t$
          \item Exogenous shocks: $\varepsilon_{t+1}$
          \item Controls: $C_t$, $K_{t+1}$
          \item Forward looking: $C_{t+1}$
        \end{itemize}
      \end{column}
      \begin{column}{0.475\textwidth}
        Equations for the equilibrium
        \begin{equation*}
          \begin{cases}
            u'(C_t) = \beta \cdot \E_t \left( u'(C_{t+1}) \left[ \alpha K_{t+1}^{\alpha-1} + 1 - \delta \right] \right) \\
            C_t + K_{t+1} = Z_t K_t^\alpha + (1 + \delta) K_t  \\
            \log(Z_{t+1}) = (1-\rho) \mu + \rho \log(Z_t) + \log(\varepsilon_{t+1})
          \end{cases}
        \end{equation*}
      \end{column}
    \end{columns}

  \end{frame}


  \begin{frame}
    \frametitle{Intro to Numerical Methods for Economists: Perturbation Methods}

    \textbf{Focus on the equations that characterize the equilibrium (w/ CRRA utility)}
    \begin{equation*}
      \begin{cases}
        C_t^{-\gamma} = \beta \cdot \E_t \left( C_{t+1}^{-\gamma} \left[ \alpha K_{t+1}^{\alpha-1} + 1 - \delta \right] \right) \\
        C_t + K_{t+1} = Z_t K_t^\alpha + (1 + \delta) K_t  \\
        \log(Z_{t+1}) = (1-\rho) \mu + \rho \log(Z_t) + \log(\varepsilon_{t+1})
      \end{cases}
    \end{equation*}

    \vfill\pause

    There exist
    \begin{itemize}
      \item A (deterministic) steady state
      \item Derivatives of each equation
    \end{itemize}

    \vfill\pause

    Log-linear representation of the model (1\textsuperscript{st} order Taylor expansion around the steady state)
    \begin{equation*}
      \begin{cases}
        c_t = \E_t ( c_{t+1} ) - \frac{1}{\gamma} \E_t \left( \alpha + (\alpha-1) k_{t+1} \right) \\
        c_t + k_{t+1} = z_t + \alpha k_t + (1+\delta) k_t \\
        z_{t+1} = \rho z_t + \log(\varepsilon_{t+1})
      \end{cases}
    \end{equation*}

    \vfill\pause

    Can solve this system of linear equations with linear algebra (e.g., Schur decomposition)

  \end{frame}

  \begin{frame}
    \frametitle{Intro to Numerical Methods for Economists: Projection Methods}

    \textbf{Focus on the optimization problem (write the associated Bellman equation)}
    \begin{align*}
      V(K, Z) = \max_{C, K'} &\; u(C) + \beta \E \left( V(K', Z') | Z \right) \\
      \text{s.t.} &\; \begin{cases}
        C + K' = Z K^\alpha + (1 + \delta) K \\
        \log(Z') = (1-\rho) \mu + \rho \log(Z) + \log(\varepsilon'), & \varepsilon' \overset{iid}{\sim} \mathcal{N}(0, \sigma^2) %\\ C, K' > 0
      \end{cases}
    \end{align*}

    \vfill\pause

    There exists
    \begin{itemize}
      \item A contraction mapping $\mathbf{T}$ induced by the ``max'' operator
      \item A unique fixed point $V(K, Z)$
    \end{itemize}

    \vfill\pause

    In a computer
    \begin{itemize}
      \item Define the domains for $K$ and $Z$
      \item Define a function that maximizes $u(C) + \beta \E (\ldots) \text{ s.t.} \ldots$
      \item Iterate the function until convergence
    \end{itemize}

    \vfill\pause

    Can crack this by letting the computer loop the contraction mapping $\mathbf{T}$

  \end{frame}

  \begin{frame}
    \frametitle{Intro to Numerical Methods for Economists: Projection Methods (cont'd)}

    For more complicated supply-demand models (let the ``real'' equilibrium price be $P^*$)
    \begin{enumerate}
      \item Guess an equilibrium price $P^{(h)}$
      \item Obtain the policy functions associated to the Bellman equation, for the given price $P^{(h)}$
      \item Define the excess demand function $D(P)$
      \item Observe that $D(P)$ is decreasing in $P$
        \begin{itemize}
          \item If $D(P^{(h)}) > 0$, then $P^{(h)} < P^*$
          \item If $D(P^{(h)}) < 0$, then $P^{(h)} > P^*$
        \end{itemize}
      \item Propose a new guess $P^{(h+1)}$ accordingly
      \item Repeat steps 1--5 until $| P^{(h+1)} - P^{(h)} | < \epsilon$
    \end{enumerate}

  \end{frame}

  \begin{frame}
    \frametitle{Intro to Numerical Methods for Economists: Takeaway's}

    We will see both projection (these classes) and perturbation methods (in Macro 4, hopefully)

    \vfill\pause

    What do we do with these numbers? E.g.
    \begin{itemize}
      \item Analyses of policy functions (if non-trivial)
      \item Impulse-Response Functions (IRFs)
      \item Counterfactual simulations
    \end{itemize}

    \vfill\pause

    Why do we need all of this?
    \begin{itemize}
      \item Can a mechanism explain macro phenomena? Write model, see variables move up/down
      \item Do these mechanisms matter quantitatively? Write model, compare simulations with data
    \end{itemize}

    \vfill\pause

    Why do projection methods matter?
    \begin{itemize}
      \item Models that suffer from derivatives (i.e., where higher order moments matter)
      \item Models with heterogeneous agents
      \item Models with binding constraits
    \end{itemize}

  \end{frame}

  \begin{frame}
    \frametitle{Intro to Python}
    % \centering

    Moving to a Jupyter Notebook

  \end{frame}

\end{document}
