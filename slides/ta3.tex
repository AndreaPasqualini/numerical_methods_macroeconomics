\documentclass[10pt, aspectratio=1610, handout]{beamer}
\usepackage{common}

\title[VFI, PFI and DP]{
  \textbf{VFI, PFI and DP in Stochastic Environments}
}

\subtitle[Macro 3: TA\#3]{
  \textbf{Macroeconomics 3:} TA class \#3
}

\author[A.~Pasqualini]{
  Andrea Pasqualini
}

\institute[Bocconi]{Bocconi University}

\date{
  22 February 2021
}

\begin{document}

  \begin{frame}
    \maketitle
  \end{frame}

  \begin{frame}
    \frametitle{Plan for Today}

    Objective: \textbf{Solve numerically for $V(K, A)$ and $K'(K, A)$}

    \vfill\pause

    Operating example: Neoclassical Growth Model (stochastic version)
    \begin{align*}
      V(K, \alert{A}) \equiv \max_{C, K'} &\; \frac{C^{1 - \gamma}}{1 - \gamma} + \beta \alert{\E} \left( V(K', \alert{A'}) \alert{| A} \right) \\
      \text{s.t.} &\;
      \begin{cases}
        C + K' \leq \alert{A} K^\alpha + (1 - \delta) K \\
        C, K' > 0 \\
        \alert{\log(A') = (1-\rho) \log(\mu) + \rho \log(A) + \epsilon} \\
        \alert{\epsilon \overset{iid}{\sim} \mathcal{N}(0, \sigma^2)}
      \end{cases}
    \end{align*}

    \vfill\pause

    \begin{columns}[T]
      \begin{column}{0.45\textwidth}
        \begin{itemize}
          \item Same methods as last time
          \item Same objects of interest
          \item Adding discretization methods for time series processes
        \end{itemize}
      \end{column}
      \begin{column}{0.45\textwidth}
        Shocks are operationally useful
        \begin{itemize}
          \item Simulations
          \item Impulse-Response Functions
          \item Forecast Error-Variance Decomposition
        \end{itemize}
      \end{column}
    \end{columns}

  \end{frame}

  \begin{frame}
    \frametitle{The Discretization Problem}

    Objective: \textbf{approximate a continuous stochastic process with a discrete one}

    \vfill\pause

    Same problem faced in our last class: the computer has no concept of set density

    \vfill\pause

    \begin{align*}
      \log(A') &= (1 - \rho) \log(\mu) + \rho \log(A) + \epsilon & \epsilon \overset{iid}{\sim} \mathcal{N}(0, \sigma^2)
    \end{align*}
    \begin{align*}
      \underbrace{\begin{bmatrix}
        \log(A_1) \\ \log(A_2) \\ \vdots \\ \log(A_m)
      \end{bmatrix}}_{\text{grid for $\log(A)$}}
      &=
      \underbrace{\begin{bmatrix}
        \Pi_{1,1} & \Pi_{1,2} & \cdots & \Pi_{1,m} \\
        \Pi_{2,1} & \Pi_{2,2} & \cdots & \Pi_{2,m} \\
        \vdots    & \vdots    & \ddots & \vdots    \\
        \Pi_{m,1} & \Pi_{m,2} & \cdots & \Pi_{m,m} \\
      \end{bmatrix}}_{\text{transition probabilities, $\Pi$}}
      \cdot
      \underbrace{\begin{bmatrix}
        \log(A_1) \\ \log(A_2) \\ \vdots \\ \log(A_m)
      \end{bmatrix}}_{\text{grid for $\log(A)$}}
    \end{align*}

    \vfill\pause

    \begin{itemize}
      \item Need to set up a grid for $A$ \dimmer{(same as before)}
      \item Need to figure out the transition probabilities \dimmer{(new!)}
        \begin{itemize}
          \item Important to compute the conditional expected continuation value $\E \left( V(K', A') | A \right)$
        \end{itemize}
    \end{itemize}

  \end{frame}

 \begin{frame}
    \frametitle{The Discretization Problem (cont'd)}

    Objective: \textbf{approximate a continuous stochastic process with a discrete one}

    \vfill

    \begin{columns}
      \begin{column}{0.35\textwidth}
        Must match
        \begin{itemize}
          \item Unconditional exp.~value
          \item Conditional exp.~value
          \item Unconditional variance
          \item Conditional variance
          \item (Optional) skewness
          \item (Optional) kurtosis
          \item (Optional) higher-order moments
        \end{itemize}
      \end{column}
      \begin{column}{0.6\textwidth}
        \begin{figure}
          \centering
          \begin{tikzpicture}
            \begin{axis}[footnotesize, width=8cm, height=7cm, domain=-3:3, xtick={-3, -2, -1, 0, 1, 2, 3}, ytick={0, 0.1, 0.2, 0.3, 0.4}, xmajorgrids=true, ymajorgrids=true, grid style=dashed, title={$\log(Z')|Z \sim \mathcal{N}(0, 1)$}]
              \addplot[samples=500, alert, very thick]{exp(-x^2) / (sqrt(2 * pi))};
              \only<2->{\addplot[ybar interval, fill=main, opacity=0.2] coordinates {(-2.5, 0.032) (-1.5, 0.183) (-0.5, 0.387) (0.5, 0.183) (1.5, 0.032) (2.5, 0)};}
            \end{axis}
          \end{tikzpicture}

          {\scriptsize\dimmer{\textbf{Note:} the parameters in this figure are chosen exclusively for illustration purposes}}
        \end{figure}
      \end{column}
    \end{columns}

  \end{frame}

  \begin{frame}
    \frametitle{Overview of Methods}

    \textbf{Tauchen}
    \begin{itemize}
      \item Constructs a histogram for a conditional distribution function
      \item Can control the grid directly
      \item Easy to code, easy intuition
      \item Approximation errors with high-persistence processes
    \end{itemize}

    \vfill\pause

    \textbf{Tauchen-Hussey}
    \begin{itemize}
      \item Constructs a histogram for a conditional distribution function
      \item Imposes a fancy grid, no control over it (except for no.~of points)
      \item Approximation errors with high-persistence processes
    \end{itemize}

    \vfill\pause

    \textbf{Rouwenhorst}
    \begin{itemize}
      \item Recursively approximates a conditional distribution function
      \item No control on grid (except for no.~of points)
      \item Robust to high-persistence processes
      \item ``It just works,'' non-obvious intuition
    \end{itemize}

  \end{frame}

  \begin{frame}
    \frametitle{The Tauchen Algorithm}

    \begin{enumerate}
      \item Forget about the unconditional average of the process \hfill \dimmer{(will recover it later)}
      \vfill\pause
      \item Create a grid for the support $S$ of the probability distribution function \hfill \dimmer{(this is a vector)}
      \vfill\pause
      \item Compute all possible transitions $S' - \rho S$ \hfill \dimmer{(this is a matrix)}
      \vfill\pause
      \item Evaluate the relevant CDF (e.g., Gaussian) at the possible transitions
      \vfill\pause
      \item Make the resulting matrix such that each row sums to one
      \vfill\pause
      \item Shift the grid by the unconditional average, if needed
    \end{enumerate}

  \end{frame}

  \begin{frame}
    \frametitle{The Tauchen-Hussey Algorithm}

    \begin{enumerate}
      \item Forget about the unconditional average of the process \hfill \dimmer{(will recover it later)}
      \vfill\pause
      \item Obtain the grid for $S$ by computing the zeros of a Gauss-Hermite polynomial of degree $m$
      \vfill\pause
      \item Rescale the grid points by $\sqrt{2\sigma^2}$ \hfill \dimmer{(this is a vector)}
      \vfill\pause
      \item Compute the relevant \textit{conditional} PDF at the possible transitions \hfill \dimmer{(this is a matrix)}
      \vfill\pause
      \item Rescale the computed conditional PDF to account for discrete points
      \vfill\pause
      \item Normalize the matrix so that rows sum to one
      \vfill\pause
      \item Shift the grid by the unconditional average, if needed
    \end{enumerate}

  \end{frame}

  \begin{frame}
    \frametitle{The Rouwenhorst Algorithm}

    \begin{enumerate}
      \item Set $p, q \in (0, 1)$
      \vfill\pause
      \item For $m=2$ grid points, construct $\Pi_2$ as
        \begin{equation*}
          \Pi_2 =
          \begin{bmatrix}
            p & 1 - p \\
            1 - q & q
          \end{bmatrix}
        \end{equation*}
      \vfill\pause
      \item For $m > 2$ grid points
        \begin{enumerate}
          \item Construct $\Pi_m$ as
            \begin{equation*}
              \Pi_m =
              p
              \begin{bmatrix}
                \Pi_{m-1} & 0 \\ 0 & 0
              \end{bmatrix}
              + (1 - p)
              \begin{bmatrix}
                0 & \Pi_{m-1} \\ 0 & 0
              \end{bmatrix}
              + (1 - q)
              \begin{bmatrix}
                0 & 0 \\ \Pi_{m-1} & 0
              \end{bmatrix}
              + q
              \begin{bmatrix}
                0 & 0 \\ 0 & \Pi_{m-1}
              \end{bmatrix}
            \end{equation*}
          \item Divide by 2 all but the top and bottom rows of $\Pi_m$ \hfill \dimmer{(those rows in the middle sum to 2)}
        \end{enumerate}
      \vfill\pause
      \item Create a grid of linearly spaced points for the support of the PDF
        \begin{enumerate}
          \item Compute $f = \sqrt{N - 1} \cdot \sigma / \sqrt{1 - \rho^2}$ \hfill \dimmer{(it relates to the uncond.~variance of the AR(1))}
          \item Create $A = \{ a_1, \ldots, a_m \}$ with $a_1 = -f$ and $a_m = f$
          \item Shift $A$ by the unconditional average, if necessary
        \end{enumerate}
    \end{enumerate}

    \vfill\pause

    \begin{itemize}
      \item Setting $p = q$ ensures homoskedasticity in the structure of shocks/innovations
      \item Setting $p = q = (1 + \rho) / 2$ matches the variance of the original AR(1) process
    \end{itemize}

  \end{frame}

  \begin{frame}
    \frametitle{Ergodic Distribution}

    Objective: \textbf{Compute the ergodic PDF of a Markov Chain}

    \vfill

    The ergodic distribution $\pi$ of a Markov Chain with transition matrix $\Pi$ is such that
    \begin{equation*}
      \begin{cases}
        \pi = \Pi' \pi \\
        \pi \iota = 1
      \end{cases}
    \end{equation*}
    where $\iota$ is a vector of 1's

    \vfill

    The system of equations above says that
    \begin{itemize}
      \item The vector $\pi$ is one eigenvector of the matrix $\Pi$\dots
      \item \dots\ in particular, the one whose elements sum to one
    \end{itemize}
    There are countless ways to compute the ergodic distribution, but this one works quite well

  \end{frame}

  \begin{frame}
    \frametitle{Calibration}

    \begin{table}
      \centering
      \begin{tabular}{clc}
        \toprule
        Symbol   & Meaning                     & Value \\
        \midrule
        \dimmer{$\alpha$} & \dimmer{Capital intensity in PF}     & \dimmer{0.30}  \\
        \dimmer{$\beta$ } & \dimmer{Discount parameter     }     & \dimmer{0.95}  \\
        \dimmer{$\gamma$} & \dimmer{CRRA parameter         }     & \dimmer{1.50}  \\
        \dimmer{$\delta$} & \dimmer{Capital depreciation   }     & \dimmer{0.10}  \\
        $\mu$    & Uncond.~avg.~of productivity   & 1.00  \\
        $\rho$   & Persistence of productivity    & 0.70  \\
        $\sigma$ & St.dev.~of productivity shocks & 0.10  \\
        \bottomrule
      \end{tabular}
    \end{table}

    \vfill

    \dimmer{The same disclaimer as in the previous class applies}

    \dimmer{$\to$ The calibration presented here is not credible in any meaningful empirical setting}

  \end{frame}

  \begin{frame}
    \frametitle{Simulation}

    Consider the necessary and sufficient conditions for the equilibrium in any model with rational expectations
    \begin{equation*}
      \E_t \left( f \left( X_{t-1}, X_t, X_{t+1} \right) \right) = 0
    \end{equation*}

    \vfill\pause

    The solution to such model is a ``policy function'' $g(\cdot)$ such that
    \begin{equation*}
      X_{t+1} = g(X_{t-1}, X_t)
    \end{equation*}
    What we call here ``policy function'' $g(\cdot)$ is a vector function containing
    \begin{itemize}
      \item The policy functions (strictly speaking) from the Bellman problem
      \item The laws of motion (e.g., the one for capital)
      \item Exogenous stochastic processes (e.g., the one for productivity)
    \end{itemize}

    \vfill\pause

    A simulation takes some initial conditions for $X_{t-1}$ and $X_t$ and applies the function $g(\cdot)$ repeatedly for a given series of shocks

  \end{frame}

  \begin{frame}
    \frametitle{Simulation (cont'd)}

    Steps to simulate from the Stochastic Neoclassical Growth Model

    \vfill\pause

    \begin{enumerate}
      \item Set a number of periods $T$ to simulate
      \vfill\pause
      \item Set $K_0$, that is the initial condition
      \vfill\pause
      \item For each $t \in \{0, \ldots, T\}$
      \vfill\pause
        \begin{enumerate}
          \item Draw a state $A_t$ from the relevant CDF
          \vfill\pause
          \item Compute current consumption and future capital holdings using the policy functions $C_t = C(K_t, A_t)$ and $K_{t+1} = K'(K_t, A_t)$
          \vfill\pause
          \item Compute all other endogenous variables using other equations of the model (e.g., production, investment)
        \end{enumerate}
    \end{enumerate}

  \end{frame}

  \begin{frame}
    \frametitle{Impulse-Response Functions}

    Objective: \textbf{Marginal effect of an exogenous shock on an endogenous variable}

    \vfill\pause

    IRFs are the marginal effects of shocks on endogenous variables predicted by the model

    \vfill\pause

    Formally, the response at horizon $h$ of variable $X_{t+h}$ to a shock (impulse) to $S_t$ is
    \begin{equation*}
      IRF_{X,S}(h) \equiv \frac{\partial X_{t+h}}{\partial S_t}
    \end{equation*}

    \vfill\pause

    IRFs are simple simulations
    \begin{itemize}
      \item The initial condition is typically the steady state of the model
      \item At time $t$, a sudden unexpected shock realizes
      \item At time $t+h$, for all $h>0$, all shocks are shut down
    \end{itemize}

  \end{frame}

  \begin{frame}
    \frametitle{Exercises}

    \begin{enumerate}
      \item Use the code for VFI/PFI I have shown in class \#2
        \begin{enumerate}
          \item Write code that, given an initial condition for the state variable, simulates the model
          \item In what sense such simulation is uninteresting?
        \end{enumerate}
      \vfill
      \item Use the code for VFI I have shown in this class
        \begin{enumerate}
          \item How would the discretization of the stochastic process work if $\rho = 0$ (i.e., the process of $A$ itself is a sequence of i.i.d.~random variables)?
          \item Code up the related discretization method and solve for the policy function
        \end{enumerate}
      \vfill
      \item Consider the code I have shown in this class for simulating the model
        \begin{enumerate}
          \item My code forces the shock to be on the grid for $A$: how would you modify the numerical policy functions (and those only!) to accomodate for any $A \in \mathbb{R}$?
          \item Code up your answer to the previous question
          \item Simulate the model for some periods (e.g., $T=250$)
          \item Compute the impulse-response functions of consumption, investment and production to a one-standard deviation shock to productivity
          \item Provide the economic intuition behind the the IRFs you have obtained
        \end{enumerate}
    \end{enumerate}

  \end{frame}

\end{document}
